\documentclass{article}
\usepackage{amsmath}

\begin{document}

\section*{Reescritura de la funcion objetivo}

Dada la funcion objetivo original:

\[ E(w_0, w_1, \ldots, w_D) = \sum_{i=1}^N \left( y_i - \left(\sum_{d=1}^D w_d \cdot x_{id} + w_0\right)\right)^2 + \lambda \sum_{d=1}^D w_d^2 + \lambda w_0^2 \]

\subsection*{1. Demostracion de la reescritura}

Para demostrar que se puede expresar como:

\[ E(\hat{w}) = \|X\hat{w} - y\|^2 + \lambda \|\hat{w}\|^2 \]

\textbf{Demostracion:}

Primero, definimos la matriz de diseño extendida (\(X\)) como:

\[ X = \begin{bmatrix}
    1 & x_{11} & x_{12} & \ldots & x_{1D} \\
    1 & x_{21} & x_{22} & \ldots & x_{2D} \\
    \vdots & \vdots & \vdots & \ddots & \vdots \\
    1 & x_{N1} & x_{N2} & \ldots & x_{ND}
\end{bmatrix} \]

donde (\(x_{ij}\)) es el elemento (\(j\))-esimo de la observacion (\(i\))-esima.

Definimos el vector de pesos extendido (\(\hat{w}\)) como:

\[ \hat{w} = \begin{bmatrix}
    w_0 \\
    w_1 \\
    w_2 \\
    \vdots \\
    w_D
\end{bmatrix} \]

La funcion objetivo original se puede expresar como:

\[ E(w_0, w_1, \ldots, w_D) = \|X\hat{w} - y\|^2 + \lambda \|\hat{w}\|^2 \]

donde (\(\| \cdot \|\)) denota la norma euclidiana.

\subsection*{2. Solucion optima para (\(\hat{w}\))}

Para encontrar el minimo de \(E(\hat{w})\), derivamos con respecto a (\(\hat{w}\)) y establecemos la derivada igual a cero:

\[ \frac{\partial E(\hat{w})}{\partial \hat{w}} = 2X^T(X\hat{w} - y) + 2\lambda \hat{w} = 0 \]

Resolviendo para (\(\hat{w}\)):

\[ \hat{w} = (X^T X + \lambda I)^{-1} X^T y \]

donde (\(I\)) es la matriz identidad de dimensiones (\(D + 1\) \times \(D + 1\)).

\subsection*{3. Hessiano de (\(E(\hat{w})\)) es definido positivo (p.d.)}

El Hessiano de (\(E(\hat{w})\)) es:

\[ H = 2X^T X + 2\lambda I \]

Para demostrar que es p.d., debemos verificar que todos sus valores propios son positivos. Dado que (\(X^T X\)) es simetrica y no negativa, y (\(\lambda > 0\)), podemos concluir que (\(H\)) es p.d.

\end{document}
